
% Thesis Abstract -----------------------------------------------------

\begin{abstractslong}        %this creates the heading for the abstract page
RoboCup is an international robotic football competition aiming at advancing research in autonomous robotics and artificial intelligence. The transition from the \(joint\) space to \(three\) dimensional space and backward is very important for balancing, walk and dynamic kicks, as well as, the need to find with great accuracy the horizon of the camera, are among the most important things in these competitions. A forward kinematics mechanism is important for the translations of the joint to Cartesian positions and an inverse kinematics mechanism is, also, important for run-time execution of dynamic trajectories. This thesis describes forward and inverse kinematics mechanisms for the humanoid robot NAO. We used the standard approach for solving the forward kinematics problem that uses the DH parameters of the joints and creates a mechanism that solves all the kinematic chains of the robot. It is possible to concatenate kinematic chains and the solutions of forward kinematics for each of these chains and create a solution for a larger chain. For inverse kinematics problem we chose to  find an analytical solution for this problem. Most of the teams that participate in RoboCup have an approximate solution for this problem, but the analytical solution is, in general, faster and doesn't have singularities. The kinematics are created and implemented for the Aldebaran NAO humanoid robot and have been integrated with the code of team ``Kouretes''.




\end{abstractslong}
\newpage
\ 


 \begin{abstractsgreeklong}
{\gr dfsafsaddsafsdaf}
 \end{abstractsgreeklong}

\newpage
\ 


% ----------------------------------------------------------------------


%%% Local Variables: 
%%% mode: latex
%%% TeX-master: "../thesis"
%%% End: 
