\chapter{Introduction}
\label{intro}
NAO is the robot that is used for the soccer games in RoboCup SPL league. In this league, all the teams are using the same robot and the same hardware, too. Thus, the teams need to implement their own omni-directional walk~\cite{naowalk}~\cite{bhumanwalk}, motions and balancing system if they want to be more competitive. The creation of dynamic movements, as the above, are not feasible without an inverse kinematics mechanism. Also, creation of a balancing system is impossible without the calculation of the center of mass using forward kinematics. This mechanism must be fast, because the RoboCup is a real-time environment. This thesis describes a forward kinematics solution for every chain of NAO and an analytical solution for the problem of inverse kinematics without any approximations.

\section{Thesis Contribution}
As it mentioned before, there is a need to know the position of an end effector and the execution of dynamic trajectories. With this thesis we succeeded to implement the mechanism that translates the joints of a chain to a Cartesian position for the end effector. Also, we created a mechanism that translates dynamic trajectories to joint values in real execution time. The contribution of this thesis to our SPL team, Kouretes, is the mechanism of inverse kinematics that makes possible the creation of our own omni-directional walk and kick engine.

\section{Thesis Outline}
Chapter~\ref{Background} describes the RoboCup competition, the Standard Platform League (SPL), our SPL team Kouretes, and the Aldebaran NAO humanoid robot. Furthermore, it provides basic background information about  generic robot kinematics, affine transformation matrices, and the Denavit-Hartenberg (DH) parameters. In Chapter~\ref{problem} we provide a complete description of the NAO hardware and we define the problem of kinematics for the NAO robot. Moreover, in Chapter~\ref{related} we discuss the related work about forward and mainly inverse kinematics for the NAO robot. In Chapter~\ref{approach} we describe in detail our solutions to the problems of forward and inverse kinematics for the NAO robot. Additionally, we explain the implementation of kinematics and integration with our team's code. In Chapter~\ref{Results} we present the real-time performance of our kinematics mechanism and a couple of scenarios to demonstrate its effectiveness. Finally, in Chapter~\ref{conclusion} we discuss the results of this thesis and we compare it with other related approaches, pointing out at the same time possible future directions.

