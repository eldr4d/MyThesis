\chapter{Introduction}
\label{intro}
NAO is the robot that is used for the soccer games in RoboCup SPL league. In this league, all the teams are using the same robot and the same hardware, too. Thus, the teams need to implement their own omni-directional walk~\cite{naowalk}~\cite{bhumanwalk}, motions and balancing system if they want to be more competitive. The creation of dynamic movements, as the above, are not feasible without an inverse kinematics mechanism. Also, creation of a balancing system is impossible without the calculation of the center of mass using forward kinematics. This mechanism must be fast, because the RoboCup is a real-time environment. This thesis describes a forward kinematics solution for every chain of NAO and an analytical solution for the problem of inverse kinematics without any approximations.

\section{Thesis Contribution}
As it mentioned before, there is a need to know the position of an end effector and the execution of dynamic trajectories. With this thesis we succeeded to implement the mechanism that translates the joints of a chain to a Cartesian position for the end effector. Also, we created a mechanism that translates dynamic trajectories to joint values in real execution time. The contribution of this thesis to our SPL team, Kouretes, is the mechanism of inverse kinematics that makes possible the creation of our own omni-directional walk and kick engine.

\section{Thesis Outline}
Chapter~\ref{Background} provides a background for the RoboCup competition and the SPL league, as well as a brief description about the Kouretes SPL team. Furthermore, it describes the affine transformation matrices, DH parameters and robot kinematics. In Chapter~\ref{problem} we provide a complete description of the hardware of NAO as well as the problem of kinematics for NAO. Moreover, in Chapter~\ref{related} we will discuss the related work about forward and mainly inverse kinematics. In Chapter~\ref{approach} we will describe analytically our solutions to the problem of forward and inverse kinematics. Also, we will explain the implementation of kinematics to fit the team code. In Chapter~\ref{Results} we will show execution times and demos about the implementation of kinematics. Finally, in Chapter~\ref{conclusion} we will discuss the results of this thesis and we will compare them with other, similar works as well as pointing out possible future directions.

