\chapter{Introduction}
\label{intro}

Articulated robots with multiple degrees of freedom, such as humanoid robots, have become popular research platforms in robotics and artificial intelligence. Such robots can perform complex motions, including balancing, walking, standing up, etc. A challenging domain, where humanoid robots are called to demonstrate complex motion skills is the RoboCup (robot soccer) competition~\cite{robocup}, whereby teams of autonomous robots compete against each other in various leagues. In this thesis we focus on the Aldebaran NAO humanoid robot, which is used exclusively by all teams competing in the Standard Platform League (SPL) of the RoboCup competition. NAO a mid-size humanoid robot with 21 degrees of freedom (independently-moving joints) divided in five kinematic chains (a head, two arms, two legs). NAO is capable of performing various complex motions and, in fact, many SPL teams have designed and implemented their own omni-directional walk algorithms~\cite{naowalk,bhumanwalk}, balancing methods~\cite{somebalancepaper}, and kick engines~\cite{bhumankickengine} to be more competitive. 

It is widely known that the design of complex dynamic motions is achievable only through the use of robot kinematics, which is an application of geometry to the study of arbitrary robotic chains~\cite{introroboticscraigbook}. Robot kinematics include forward and inverse kinematics. The forward kinematics provide the means to map any configuration of the robot from its own multi-dimensional joint space to the three-dimensional physical space in which the robot operates, whereas the inverse kinematics provide the means to finding joint configurations that drive the end effectors of the robot to desired points in the three-dimensional space. It is easy to see why kinematics are required in any kind of complex motion design. Stable walk gaits rely on the ability of the robot to follow planned trajectories with its feet; this is not possible without some mechanism that allows the robot to set its joints to angles that drive the feet to points along such trajectories, an instance of inverse kinematics. Likewise, balancing methods rely on the ability to calculate the center of mass of the robot, which is constantly changing as the robot moves; finding the center of mass is made possible, only if the exact position and orientation of each part of the robot in the three-dimensional space is known, an instance of forward kinematics. It is also quite understandable that any kinematics computations must be performed in real-time to be useful in dynamic motions. 


\section{Thesis Contribution}
This thesis studies the problems of forward and inverse kinematics for the Aldebaran NAO humanoid robot and contributes for the first time a complete analytical solution to both problems with no approximations. In addition, it contributes an implementation of the proposed NAO kinematics as a software library for real-time execution on the robot. This work enables NAO software developers to make transformations between configurations in the joint space and points in the three-dimensional physical space in just milliseconds. 

The proposed solution was made possible through a decomposition into five independent problems (head, two arms, two legs), the use of the Denavit-Hartenberg method~\cite{dhparam1,dhparam2}, and the analytical solution of a non-linear system of equations. Existing methods for NAO inverse kinematics either offer analytical solutions~\cite{bhuman}, but only under certain simplification assumptions, or offer approximate numerical solutions~\cite{naopaper}, which are nevertheless subject to singularities. The main advantage of the proposed inverse kinematics compared to existing approaches is its accuracy, its efficiency, and the elimination of assumptions and singularities.   

This thesis additionally contributes two demonstrations of NAO kinematics: (a) a pointing-to-the-ball task, whereby the robot tracks a ball in the field and uses inverse kinematics to point to the exact location of the ball with its extended arm(s), and (b) a simple balancing method, whereby the robot calculates its current center of mass through the help of forward kinematics and drives one of its legs to the projection of the center of mass on the floor using inverse kinematics to maintain balance. The implemented NAO kinematics library has been integrated into the software architecture of the RoboCup team ``Kouretes'' and is currently being used in various motion design problems, such as dynamic balancing, trajectory following, dynamic kicking, and omnidirectional walking.

\section{Thesis Outline}
Chapter~\ref{Background} describes the RoboCup competition, the Standard Platform League (SPL), our SPL team Kouretes, and the Aldebaran NAO humanoid robot. Furthermore, it provides basic background information about  generic robot kinematics, affine transformation matrices, and the Denavit-Hartenberg (DH) parameters. In Chapter~\ref{problem} we provide a complete description of the NAO hardware and we define the problem of kinematics for the NAO robot. Moreover, in Chapter~\ref{related} we discuss the related work about forward and mainly inverse kinematics for the NAO robot. In Chapter~\ref{approach} we describe in detail our solutions to the problems of forward and inverse kinematics for the NAO robot. Additionally, we explain the implementation of kinematics and integration with our team's code. In Chapter~\ref{Results} we present the real-time performance of our kinematics mechanism and a couple of scenarios to demonstrate its effectiveness. Finally, in Chapter~\ref{conclusion} we discuss the results of this thesis and we compare it with other related approaches, pointing out at the same time possible future research directions.

