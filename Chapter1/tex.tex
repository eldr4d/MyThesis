\chapter{Introduction}
\label{intro}
NAO is the robot that is using for the soccer games in RoboCup SPL league. In this league all the teams are using the same robot, so the same hardware. Thus, the teams need to implement there own walk, motions and balancing system if they want to be more competitive. The creation of dynamical movements, as the above, are not feasible without an inverse kinematic mechanism. Also, the balancing is impossible without the calculation of the center of mass with forward kinematics. This mechanism must be fast, because the RoboCup is a real-time environment. This thesis describes a forward kinematics solution for every chain of the NAO and an analytical solution for the problem of inverse kinematics without any approximations.

\section{Thesis Constribution}
As it mentioned before, there is a need to know the position of an end effector and the execution of dynamic trajectories. With this thesis we succeed to make a the mechanism that translate the joints of a chain to a Cartesian position for the end effector. Also we created a mechanism that translate dynamic trajectories to joint values in real-time. The contribution of this thesis to our SPL team, Kouretes, is the mechanism of inverse kinematics that make possible the creation of our walk and our kick engine.

\section{Thesis Outline}
Chapter~\ref{Background} provides a background for the RoboCup competition and the SPL league as long as brief description of the Kouretes SPL team. Furthermore, it describes the affine transformation matrix, DH parameters and robot kinematics. In Chapter~\ref{problem} we provide a complete description of the hardware of NAO as long as the problem of kinematics for NAO. Next in Chapter~\ref{related} we will discuss the related work about kinematics. In Chapter~\ref{approach} we will describe analytical our solutions to the problem of forward and inverse kinematics. Chapter~\ref{Implementation} explains the implementation of kinematics to the team code. In Chapter~\ref{Results} we will show demos about the implementation of kinematics. Chapter~\ref{future} has some interesting ideas for the use of kinematics for the future. Finally in Chapter~\ref{conclusion} we will discuss the results of this thesis and we will compare them with other, similar, works

