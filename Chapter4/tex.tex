\chapter{Related Work}
\label{related}
The problem of forward and inverse kinematics is a familiar problem for all the teams that participate in the RoboCup SPL. The solution to the problem of forward kinematics is very easy and all the teams have implemented their own. There are not too many known solutions for the problem of inverse kinematics for NAO robot. On the other hand, Aldebaran provides an approximate solution for this problem but as we will see below we can't use this solution in our approach. Also, some teams have published their own analytical solutions.
\section{Aldebaran Forward and Inverse Kinematics Solution}
\textbf{Aldebaran Forward Kinematics Solution}\\
Aldebaran provides a forward kinematics mechanism but the problem is that it provides the solution only for the current state of the robot. So you can't find the position of the camera at the time a specific picture was taken given the joints values. Also, as we said before, we must find the solution for this problem if we want to solve inverse kinematics problem. However, Aldebaran provides us with the DH parameters for all the joints of the robot and that was very useful.\\ 
\textbf{Aldebaran Inverse Kinematics Solution}\\
Aldebaran has implemented in the API of the robot some functions that move an end effector to a given point in the \(3-dimensional\) space. This functions are using the Jacobian approximate method to find the solution to the problem of inverse kinematics. The omni-directional walk that Aldebaran provide us with, uses this solution to execute the trajectories. Although this solution is accurate, it can easily fall into a singularity and if this happens, it will stuck in there. That is a very bad problem because then the whole motion of the robot gets stuck.
\section{BHuman Inverse Kinematics Solution}
B-Human is a RoboCup SPL team from the university of Bremen in Germany. Every year they publish the code release~\cite{bhuman}, the code that they used in the last RoboCup and a documentation for this code. In the code release they have an inverse kinematics solution for the legs of NAO but with an approximation. The solution provided, always makes the end effector parallel to the plane of the torso, that is defined by the \(z\) and \(x\) axis. So this approximation supplies us with a solution that moves the end effector to the target point but with different orientation from the target orientation.
\section{QIAU Inverse Kinematics Solution}
MRL SPL team is a team from the university of QIAU in Teheran. They have published a paper in which they present a solution that theoretically solves the problem of inverse kinematics for the legs. We have tried to implement this solution, but the results were not satisfactory.