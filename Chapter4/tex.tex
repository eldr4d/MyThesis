\chapter{Related Work}
\label{related}

The problem of forward and inverse kinematics for the NAO robot is a familiar problem to all the teams participating in RoboCup SPL. The solution to forward kinematics is quite straightforward and most teams have implemented their own code for forward kinematics computations. There are only a few known solutions for the problem of inverse kinematics. We review existing related work in the next few sections. 

\section{Aldebaran Robotics}

\subsubsection*{Aldebaran Forward Kinematics Solution}
Aldebaran Robotics provides a forward kinematics mechanism integrated within the proprietary NaoQi middleware for the NAO robot. However, this mechanism does not accept any input and provides a solution only for the current joint configuration of the robot. As a result, it is impossible to run Aldebaran's forward kinematics for a specific set of joint values, for example the joint values recorded when a specific picture was acquired from the camera. The ability to provide any set of joints is important, not only for finding the position of the camera in the three-dimensional at specific times, but also for verifying candidate solutions returned by inverse kinematics. On the other hand, Aldebaran provides the DH parameters for all the joints of the NAO robot and that was very useful.

\subsubsection*{Aldebaran Inverse Kinematics Solution}
Aldebaran Robotics provides an inverse kinematics mechanism integrated within the proprietary NaoQi middleware for the NAO robot. These functions in the API of the robot can move the end effector of a kinematic chain to a given point in the three-dimensional space. The method used to provide the solution is based on the Jacobian iterative approximation method. Furthermore, the omni-directional walk engine provided by Aldebaran Robotics uses this approach to follow planned foot trajectories. Although the resulting solutions are in most cases accurate, the method can easily fall into singularities; if that happens, the robot gets stuck in a specific configuration. Singularities present a serious problem with possible catastrophic consequences for the robot. 

\section{B-Human Inverse Kinematics Solution}
B-Human is the RoboCup SPL team of the University of Bremen in Germany. Each year they publish a code release, which includes the full code they used in the last RoboCup and a documentation for this code. In their recent code release~\cite{bhuman} they include an inverse kinematics solution for the legs of NAO, albeit with under certain simplification assumptions and approximations. The solution provided always makes the foot parallel to the plane defined by the $z$-axis and the $x$-axis of the torso. If the target point violates this assumption, the solution will reach the target position, but will ignore the target orientation, and therefore it will only be an approximate solution. 

\section{QIAU Inverse Kinematics Solution}
MRL is the RoboCup SPL team of the Qazvin Islamic Azad University (QIAU) in Tehran, Iran. They have published~\cite{iran} an analytical solution for the problem of inverse kinematics for the legs. We have tried to implement their solution, but unfortunately we were not able to reproduce their results.