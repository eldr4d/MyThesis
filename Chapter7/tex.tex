\chapter{Conclusion}
\label{conclusion}

Kinematics is the base for several applications related to robot motion. Our approach to NAO kinematics is based on standard principled methods for studying robot kinematic chains. Nevertheless, no complete analytical solution for the NAO robot had been published before. Our work offers such a complete analytical solution, which we expect will be useful not only to RoboCup SPL teams, but also to any NAO software developer. 


\section{Future Work}
The work in this thesis can be used as the base for a several future research directions, some of which are listed below. It is also a step towards making the software architecture of our team independent from Aldebaran's development framework.

\subsubsection*{Omni-Directional Walk}

The ability of a robot to walk towards any desired direction is called omni-directional walk. The feet (and arms) trajectories in omni-directional walk engines are being calculated dynamically and for this reason an inverse kinematics mechanism is more than necessary for the robot to be able to follow them. 

\subsubsection*{Dynamic Balancing}

As the robot walks, kicks, or performs any kind of motion, it must maintain its balance. Knowledge of the position of the center of mass at all times is more than necessary for successful balancing.

\subsubsection*{Kick Engine}

Currently, our team relies  only on static predefined kicks designed. The problem with those kicks is that they cannot absorb random disturbances and quite often robots executing them fall down. With inverse kinematics and balancing in place, the team can develop a dynamic kick engine, which takes care of balancing the robot on one leg, while following a dynamic kick trajectory based on the ball's position with the other leg.